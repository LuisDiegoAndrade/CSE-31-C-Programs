\documentclass{article}%
\usepackage[T1]{fontenc}%
\usepackage[utf8]{inputenc}%
\usepackage{lmodern}%
\usepackage{textcomp}%
\usepackage{lastpage}%
\usepackage{amsmath}%
%
%
%
\begin{document}%
\normalsize%
\section{TPS Activity 2}%
\label{sec:TPSActivity2}%
1) gcc {-}g {-}o execName sourcename.c%
\[%
\linebreak%
\]%
2) lldb execName%
\[%
\linebreak%
\]%
3) Shortcut "r"%
\[%
\linebreak%
\]%
4) Lets you stop the flow of exection at a particular line. In lldb use "b sourcefile.c:linenumber"%
\[%
\linebreak%
\]%
5) Using shortcut "n"%
\[%
\linebreak%
\]%
6) Using "p VARIABLENAME"%
\[%
\linebreak%
\]%
7) Using "continue"%
\[%
\linebreak%
\]%
8) Using "exit"

%
\section{TPS Activity 3}%
\label{sec:TPSActivity3}%
1) 4%
\[%
\linebreak%
\]%
2) x, y, and arr{[}0{]} have all been declared, but not initialized. Most likely there might be something stored at the memory allocated for these variables.%
\[%
\linebreak%
\]%
3) By initializing the variables.%
\[%
\linebreak%
\]%
8) Yes because they are both pointing to the same address in memory%
\[%
\pagebreak%
\]

%
\section{Assignment 1}%
\label{sec:Assignment1}%
1) 4%
\[%
\linebreak%
\]%
2) Line 15, scanf() is given an incorrect parameter. scanf() needs the variable input address. E.g scanf("\%d", \&input)%
\[%
\linebreak%
\]%
3) The function read\_values() is not manipulating the variable sum defined in main.%
\[%
\linebreak%
\]%
4) We can pass in sums address to the function, so a pointer within this function can manipulate sums value.%
\[%
\linebreak%
\]

%
\section{Assignment 2}%
\label{sec:Assignment2}%
We need to allocate enough memory in str1 to be able to concate both strings without truncating any values.%
\[%
\linebreak%
\]

%
\end{document}
